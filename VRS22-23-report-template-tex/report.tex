\documentclass[a4paper]{article}
\usepackage{macros-ohp}
\definecolor{vrscolor}{RGB}{255, 255, 255}



%Please make sure the tex is compiled twice to have all the background images displayed correctly.

\title{\Huge \textbf{ \textcolor{vrscolor}{ \textless RELATÓRIO DE GESTÃO 2019 \textgreater}}}

\author{\Huge \textless Mário Diego Rocha Valente\textgreater\\
	\Large  \textless Analista de Trânsito Matrícula: 57195478-1 \textgreater \\
	\Large \textless DETRAN-PA \textgreater\\
}
\date{}

\linespread{1.5}

\begin{document}




\begin{titlingpage}
\tikz[remember picture,overlay] \node[opacity=1,inner sep=0pt] at (current page.center){\includegraphics[width=\paperwidth,height=\paperheight]{imgs/background.png}};
\vspace*{3.5cm}
{\let\newpage\relax\maketitle}
\vspace*{\fill}
\end{titlingpage}


\tableofcontents



\section{APRESENTAÇÃO}

Apresenta-se Relatório de Monitoramento da Execução das Ações de Educação para o Trânsito, constantes do Plano Plurianual do Departamento de Trânsito do Estado do Pará, referente ao acumulado até segundo quadrimestre de 2019 (janeiro a dezembro), informando o andamento das ações, com objetivo de avaliação da necessidade de interferência da gestão, primando pelo pleno cumprimento dos objetivos programados.\vskip0.3cm

Nas tabelas foram consideradas a previsão orçamentária atualizada para análise da execução orçamentária e financeira, com base no Orçamento Geral do Estado definido no Plano Plurianual.\vskip0.3cm

Para a elaboração desse trabalho e melhor visualização dos resultados, a equipe extraiu dados do sistema SIGPLAN – SEPLAN/PA e Relatórios Mensais Gerenciais (janeiro a dezembro de 2019), podendo sofrer alterações futuras conforme inserção de novos dados no sistema.\vskip0.3cm




\section{Aspectos Gerais}

Criado em dezembro de 1972 pela Lei n.º 4.444, o Departamento de Trânsito do Estado do Pará- DETRAN/PA surgiu em forma de autarquia com autonomia técnica administrativa, financeira e patrimonial, visando cuidar do Trânsito no Estado do Pará. O Departamento de Trânsito do Estado do Pará é órgão executivo integrado ao Sistema Nacional de Trânsito e ao Sistema de Segurança Pública do Estado, além de vinculado à Secretaria Executiva de Segurança Pública e Defesa Social do Pará.\vskip0.3cm

O Sistema Nacional de Trânsito - SNT é composto pelos órgãos normativos, consultivos, executivos e rodoviários nas diferentes esferas de governo, tal organização pode ser visualizada na figura 1.\vskip0.3cm


\section{Organização Administrativa}


A reorganização do órgão foi definida pela Lei n° 7.594, de 28 de dezembro de 2011, que começou a vigorar a partir da publicação ocorrida em 29 de dezembro do mesmo ano, da qual ainda não foi concluído o novo Regimento Interno.\vskip0.3cm

Art. 2°. São funções básicas do Departamento do Estado do Pará:\vskip0.3cm

\begin{enumerate}

\item Cumprir e fazer cumprir a legislação e as normas de trânsito, no âmbito de suas atribuições;
\item Realizar, fiscalizar e controlar o processo de formação e reciclagem de condutores, expedir permissão para dirigir, expedir e cassar licença de aprendizagem, autorização para conduzir ciclomotores e Carteira Nacional de Habilitação;
\item Vistoriar, registrar, emplacar, selar a placa, e licenciar veículos, expedindo Certificado de Registro de Veículos – CRV e Certificado de Registro e Licenciamento de Veículos – CRLV;
\item Estabelecer, em conjunto com a Polícia Militar, as diretrizes para o policiamento ostensivo de trânsito;
\item Executar a fiscalização de trânsito, autuar e aplicar as penalidades por infrações e medidas administrativas cabíveis previstas nos artigos 21 e 22 do CTB nas áreas urbana e rural;
\item Supervisionar o controle de aprendizagem para conduzir veículos automotores;
\item Fiscalizar o nível de emissão de poluentes e ruídos produzidos pelos veículos automotores ou pela sua carga, além de dar apoio às ações específicas dos órgãos ambientais locais quando solicitado;
\item Coletar dados estatísticos e elaborar estudos sobre acidentes de trânsito e suas causas;
\item Arrecadar valores provenientes de estada e remoção de veículos e objetos;
\item Articular-se com os demais órgãos do Sistema Nacional de Trânsito no Estado, sob coordenação do respectivo CETRAN;
\item Emitir Autorização Especial de Trânsito – AET;
Parágrafo único. No exercício de sua missão, o Departamento de Trânsito do Estado do Pará – DETRAN/PA poderá celebrar convênios com órgãos executivos de trânsito nos municípios integrados ao Sistema Nacional de Trânsito no Estado do Pará, com vistas ao fornecimento de dados cadastrais dos veículos registrados e dos condutores habilitados, para fins de imposição e notificação de penalidades e de arrecadação de multas nas áreas de suas competências.
\end{enumerate}





\section{Competencias Administrativa}

O Departamento de Trânsito do Estado do pará possui varias diretorias, uma delas e a Diretoria Técnica e Operacional (DTO), que se divide em um tripé muito importante:

\begin{itemize}
    \item Educação;
    \item Engenharia;
    \item Fiscalização
\end{itemize}


A missão da coordenadoria de Educação de Trânsito é desenvolver ações que despertem tanto nos condutores quando nos pedestres atitudes mais seguras no trânsito. Estas ações assumem o formato de campanhas educativas, palestras informativas e cursos de capacitação para os interessados em contribuir com um trânsito seguro e cidadão.\vskip0.3cm

Para que este trabalho seja desenvolvido de forma mais eficaz em todo o Estado do Pará em seus 144 municípios, a Coordenadoria de Educação dividi-se em 4 gerências e suas respectivas especialidades.\vskip0.3cm 


\subsection{Gerência de Cultuta de Trânsito}
À gerência de cultura de trânsito, diretamente subordinada à Coordenadoria de Educação do DETRAN-PA, compete:
\vskip0.3cm



\begin{itemize}
\item Planejar, elaborar e executar o desenvolvimento de ações intersetoriais educativas de trânsito realizadas junto às instituições públicas privadas, organizações internacionais, não governamentais, entidades e empresas;
\item Elaborar e executar programas e projetos relacionados a promoção de um nova cultura de trânsito que valorize a qualidade de vida, a promoção da paz e respeito ao  meio ambiente;
\item Fomentar percerias com a sociedade civil organizada a fim de disseminar hábitos e comportamentos seguros no trânsito para a redução dos riscos de acidnetes;
\item Integrar-se às instituições educacionais públicas e privadas de ensino básico e superior, através de ações interdisciplinares, para a promoção de novos valores sociais a partir do tema trânsito;
\item Mobilizar voluntários sociais para a atuação em campanhas de prevenção de acidentes e segurança no trânsito com a valorização da cidadania ativa;
\item Participar de eventos, feiras e espaçoas culturais disseminando a cultura  de segurança no trânsito;
\item Desenvolver metodologias de trabalho específicas para realizar ações culturais relacionadas ao comportamento seguro no trânsito;
\item Articular-se às ações das políticas públicas que estejam relacionados à educação  e promoção da ética e da cidadania;
\item Apresentar proposta ao setor de comunicação a divulgação na mídia e nos diversos meios de comunicação de informações sobre projetos e ações desenvolvidas sobre a culura de trânsito seguro e responsável;
\item Promover parcerias com instituições de ensino superior para realizar pesquisas sociológicas, antropológicas, jurídicas e outras correlatas a area de tânsito enquanto fenômeno social;
\end{itemize}




\subsection{Gerência de Integração Educacional}
À gerência de integração Educacional, diretamente subordinada à Coordenadoria de Educação do DETRAN-PA, compete:
\vskip0.3cm 

\begin{itemize}
\item Planejar, realizar e monitorar trabalho de percusão visando a promoção de parcerias com os órgãos e entidades que estejam relacionados com os programas, projetos e ações educativas da CED;
\item Orientar e acompanhar o desenvolvimento de ações relizadas pelas instituições governamentais e não governamentais parceiras do Programa de Educação do DETRAN-PA;
\item Propor e fomentar parcerias que visem à implementação, a manutenção e ao aperfeiçoamento de ações ligadas às diversas políticas, com vistas a garantir a visão holística das ações da CED;
\item Fomentar, junto aso parceiros, a capacitação e formação de agentes multiplicadores em arte-educação;
\item Integrar-se aos projetos de educação e segurança nas escolas, desenvolvidas pelo Sistema Estadual de Segurança Pública e Sistema Ensino Público e Privado;
\item Integrar-se aos projetos e programas desenvolvidos na esfera federal, estadual, municipal, nas empresas e ONG’S promovendo ações educativas de trânsito seguro;
\item Disseminar entre os setores do DETRAN-PA a prticipação dos servidores em campanhas educativas internas e nas atividades promovidas pelo òrgão;
\item Subsidiar as CED com informações específicas das atividades pertinentes a sua area de atuação;
\item Propor a CED o aperfeiçoamento dos procedimentos internos, visando a melhoria contínua dos trabalhos desenvolvidos;
\end{itemize}


\subsection{Gerência de Programas e Projetos Pedagógicos}

À gerência de Programas e Projetos Pedagógicos, diretamente subordinada à Coordenadoria de Educação do DETRAN-PA, compete:
\vskip0.3cm 

\begin{enumerate}
\item Planejar e elaborar projetos pedagógicos para dar suporte às ações das demais gerências da CED;
\item Articular-se em rede aos programas e projetos desenvolvidos por outros órgãos do Governo do Estado como forma de ampliar intersetorialmente as ações de Educação de Trânsito;
\item Avaliar e supervisionar os programas, projetos e ações pedagógicas da Educação de Trânsito;
\item Desenvolver metodologias específicas para os programas  e projetos pedagógicos;
\item Fornecer subsídios para o planejamento das ações a serem desenvolvidas pela CED;
\item Participar das operações programadas pelo DETRA-PA, em parceria com a engenharia, fiscalização e outros setores afins;
\item Orientar, acompanhar e supervisionar o desenvolvimento de ações relizadas pelas instituições governamentais e não governamentais parceiras da CED;
\item Subsidiar a CED com informações específicas das atividades pertinentes à sua área de atuação;
\item Propor a CED o aperfeiçoamento das ações, visando à melhoria e a qualidade do atendimento às demandas;
\end{enumerate}



\subsection{Gerência de Escola Pública de Trânsito}

À gerência de Escola Pública de Trânsito, diretamente subordinada à Coordenadoria de Educação do DETRAN-PA, compete:

\begin{enumerate}
\item Planejar, elaborar, executar e acompanhar o desenvolvimento de projetos, cursos e ações educativas de trânsito voltados ao exercício da cidadania;
\item Planejar e executar cursos de capacitação para agentes multiplicadores em educação de Trânsito de acordo com os programas e projetos desenvolvidos no DETRAN-PA;
\item Propor a realização de parcerias com outros órgãos e instituições governamentais para realizar cursos de reciclagem de condutores e de aperfeiçoamento para as categorias do trânsito;
\item Planejar, executar e monitorar curso teórico de 1ª habilitação para alunos do ensino médio de acordo com as diretrizes da Política Nacional de Trânsito (PNT) e Resoluções do CONTRAN;
\item Planejar, executar e acompanhar cursos de formação especializada para categorias profissionais do trânsito e outros cursos de acordo com Resolução do CONTRAN e Legislação vigente;
\item Propor o desenvolvimento de metodologias para a formação continuada da equipe da CED;
\item Elaborar o projeto pedagógico de acordo com a Lei de Diretrizes e Bases da Educação Nacional (LDB), dos Parâmetros Curriculares Nacional (PCN), das Diretrizes do DENATRAN e os objetivos e diretrizes da Política Nacional de Trânsito;
\item Articular-se às instituições educacionais públicas e privadas de ensino superior, a fim de desenvolver estudos, pesquisas e produzir conhecimentos interdisciplinares sobre a temática trânsito;
\item Promover a avaliação das ações de educação de trânsito através da coleta de dados e realização de pesquisas.
\end{enumerate}

























\end{document}